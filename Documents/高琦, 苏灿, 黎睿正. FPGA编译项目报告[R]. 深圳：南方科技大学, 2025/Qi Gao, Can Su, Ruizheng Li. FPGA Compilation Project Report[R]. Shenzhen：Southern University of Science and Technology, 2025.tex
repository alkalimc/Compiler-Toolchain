% !Mode:: "TeX:UTF-8"
% !TEX program  = xelatex
% !BIB program  = biber
\documentclass[AutoFakeBold,AutoFakeSlant,language=chinese,degree=bachelor]{sustechthesis}
\usepackage{pgfplots}
\usepackage{booktabs}
\usepackage{caption}
\usepackage{listings}
\usepackage{xcolor}
\lstset{
  basicstyle=\ttfamily\small,
  keywordstyle=\color{blue},
  commentstyle=\color{gray},
  stringstyle=\color{red},
  breaklines=true,
  frame=single,
  columns=fullflexible
}
\input{config/preamble.tex}
% !Mode:: "TeX:UTF-8"
% !TEX program  = xelatex
\设置信息{
    % 键 = {{中文值}, {英文值}},
    分类号 = {{}, {}},
    编号 = {{}, {}},
    UDC = {{}, {}},
    密级 = {{}, {}},
    % 仅题目(不含副标题)、系别、专业,支持手动 \\ 换行,不支持自动换行。
    题目 = {{FPGA编译项目报告}, {}},
    % 如无需副标题,删除值内容即可,不可删除键定义。
    副标题 = {{技术报告}, {}},
    姓名 = {{高琦,苏灿,黎睿正}, {Qi Gao, Su Can, Li Ruizheng}},
    学号 = {{12413316}, {12413316}},
    系别 = {{}, {}},
    专业 = {{}, {}},
    指导老师 = {{余浩}, {}},
    时间 = {{}, {}},
    职称 = {{}, {}},
}


\begin{document}
\前序格式化
\摘要标题
\正文格式化

\section{项目背景}
目前大模型部署存在的难点
\begin{itemize}
    \item 工具结合度低
    
        现有的量化、打分和部署等工具模块间甚至模块内操作逻辑不一致,许多操作之间仍需手动调整参数。
    \item 新模型适配成本高
    
        现有的工作流需要根据所使用的模型手动调整,新模型需要逐一环节调试,无法开箱即用。
    \item 操作学习成本高
    
        现有的工具多为命令行工具,缺少直观的交互逻辑和图形化界面,用户操作上手难度较大。
\end{itemize}

\section{项目内容}
\subsection{技术特性}
\begin{itemize}
    \item 量化:支持GPTQ指定bits、group\_size、desc\_act
    \item 打分
        \begin{itemize}
            \item 支持lm-evaluation-harness测试项
            
                arc\_easy、arc\_challenge、gsm8k\_cot、gsm8k\_platinum\_cot、hellaswag、mmlu、gpqa、boolq、openbookqa
            \item 支持EvalPlus测试项
            
                humaneval、mbpp
        \end{itemize}

    \item GPU部署:支持vLLM指定上下文长度、显存占用限制、服务端口、API密钥
    \item 权重处理:支持Compiler-VCU128指定bits、group\_size、desc\_act
    \item FPGA服务:支持Fast API指定上下文长度、生成温度、服务端口、API密钥
    \item GPU、Port调度器:支持GPU设备图分类、调整调度显存占用量、设备锁机制
    \item WebUI:支持指定服务地址、用户管理
\end{itemize}

\subsection{WebUI}
前端WebUI主要支持用户进行一键式操作,为大模型上板做好准备工作,提供用户登录、各项部署细节选择、取消进程、实时日志查看服务。
\begin{itemize}
    \item 网页界面展示
        \begin{figure}[htb]
            \centering
            \begin{subfigure}[t]{.3\textwidth}
                \centering
                \includegraphics[width=\textwidth]{./figures/login_display.png}
                \caption{用户登录界面展示}
            \end{subfigure}
            \hfill
            \begin{subfigure}[t]{.48\textwidth}
                \centering
                \includegraphics[width=\textwidth]{./figures/log_display.png}
                \caption{日志界面展示}
            \end{subfigure}
            \hfill
            \begin{subfigure}[t]{.2\textwidth}
                \centering
                \includegraphics[width=\textwidth]{./figures/deploymenttool_display.png}
                \caption{部署工具界面展示}
            \end{subfigure}
            \caption{WebUI}
        \end{figure}
        \begin{lstlisting}[language=HTML]
        //工具选项设置代码
    data() {
        return {
          activeSections: ['model', 'quant', 'target', 'eval', 'task'],
          selectedModel: '',
          selectedQuantPrecision: 'int4',
          isDeploying: false,
          deployStatus: [],
          pollingInterval: null,
          progressPollingInterval: null,
          quantPid: null,
          selectedEvalMethod: 'evalPlus',
          selectedEvalTarget: 'none',
          quantLogs: [],
          evalLogs:[],
          dataLogs:[],
          models: [
            { value: 'qwen2', label: 'Qwen2-7B-Instruct', icon: modelQwen },
            { value: 'qwen2.5', label: 'Qwen2.5-7B-Instruct', icon: modelQwen },
            { value: 'qwen2-vl', label: 'Qwen2-VL-7B-Instruct', icon: modelQwen },
            { value: 'qwen2.5-vl', label: 'Qwen2.5-VL-7B-Instruct', icon: modelQwen },
            { value: 'deepseek', label: 'DeepSeek-R1-Distill-Qwen-7B', icon: modelDeepseek }
          ],
          precisions: [
            { value: 'int2', label: 'INT2', precisionValue: 2 },
            { value: 'int4', label: 'INT4(仅支持)', precisionValue: 4 },
            { value: 'int8', label: 'INT8', precisionValue: 8 }
          ],
          evalMethods: [
            { label: 'EvalPlus', value: 'evalPlus' },
            { label: 'lmEvaluationHarness', value: 'lmEvalHarness' }
          ],
          evalPlusTasks: [
            { value: 'humaneval', label: 'HumanEval' },
            { value: 'mbpp', label: 'MBPP' }
          ],
          lmEvalHarnessTasks: [
            { value: 'arc_easy', label: 'ARC Easy' },
            { value: 'arc_challenge', label: 'ARC Challenge' },
            { value: 'gsm8k_cot', label: 'GSM8K CoT' },
            { value: 'gsm8k_platinum_cot', label: 'GSM8K Platinum CoT' },
            { value: 'hellaswag', label: 'HellaSwag' },
            { value: 'mmlu', label: 'MMLU' },
            { value: 'gpqa', label: 'GPQA' },
            { value: 'boolq', label: 'BoolQ' },
            { value: 'openbookqa', label: 'OpenBookQA' }
          ],
          selectedEvalTasks: [],
          evalTargets: [
            { label: '原模型', value: 'origin' },
            { label: '量化模型', value: 'quant' },
            { label: '两个都评分', value: 'both' },
            { label: '不评分', value: 'none' }
          ],
        }
    },
        \end{lstlisting}
        
    \item 用户管理:支持自定义登录用户
    \begin{lstlisting}[language=HTML]
    //前端登录界面代码
    <script>
    import axios from 'axios'
    
    export default {
      name: 'Login',
      data() {
        return {
          loading: false,
          loginForm: {
            username: '',
            password: ''
          },
          rules: {
            username: [
              { required: true, message: '请输入用户名', trigger: 'blur' }
            ],
            password: [
              { required: true, message: '请输入密码', trigger: 'blur' }
            ]
          }
        }
      },
      methods: {
        handleLogin() {
          this.$refs.loginForm.validate(valid => {
            if (valid) {
              this.loading = true
              this.verifyCredentials()
            }
          })
        },
        async verifyCredentials() {
          try {
            const authHeader = 'Basic ' + btoa(`${this.loginForm.username}:${this.loginForm.password}`)
            const response = await axios.get('/api/verify', {
              headers: { 'Authorization': authHeader }
            })
    
            if (response.data.success) {
              this.$emit('login-success', {
                username: this.loginForm.username,
                password: this.loginForm.password
              })
            } else {
              throw new Error(response.data.message || '认证失败')
            }
          } catch (error) {
            let errorMsg = '登录失败: '
            if (error.response) {
              errorMsg += error.response.data?.message || error.response.statusText
            } else {
              errorMsg += error.message
            }
            this.$message.error(errorMsg)
            console.error('登录错误详情:', error)
          } finally {
            this.loading = false
          }
        }
      }
    }
    </script>
    \end{lstlisting}
    提供用户登录界面,使用体验更规范,同时可根据不同账户指定特定权限如管理员权限。
    支持客户添加指定账户,后续可拓展使用实时token验证方式提高安全性。

    \item 打分:支持同时开展多框架多测试项测试和原模型、量化模型对比测试
    \begin{lstlisting}[language=HTML]
    //轮询日志同步进度信息
    async startEvaluationPolling(target) {
      let failCount = 0;
      const MAX_FAILS = 5;

      const interval = setInterval(async () => {
        try {
          const response = await axios.get(`${this.apiUrl}/eval_progress`, {
            headers: {
              'Authorization': 'Basic ' + btoa(`${this.authInfo.username}:${this.authInfo.password}`)
            }
          });

          if (response.data.success) {
            failCount = 0;
            const progressLines = response.data.progress || [];
            this.evalLogs.push(...progressLines);
            this.$emit('eval-log', progressLines);

            const hasError = this.evalLogs.some(line =>
                line.includes('[ERROR]') ||
                line.includes('失败') ||
                line.includes('异常') ||
                line.includes('Traceback')
            );

            const hasCompleted = this.evalLogs.some(line =>
                line.includes('完成') ||
                line.toLowerCase().includes('evaluation finished') ||
                line.toLowerCase().includes('scoring complete') ||
                line.toLowerCase().includes('done')
            );

            if (hasError) {
              clearInterval(interval);
              this.deployStatus.push(`❌ ${target === 'origin' ? '原模型' : '量化模型'} 评分失败,请检查日志`);
              return;
            }

            if (!response.data.is_running && !hasCompleted) {
              clearInterval(interval);
              this.deployStatus.push(`❌ ${target === 'origin' ? '原模型' : '量化模型'} 评分中断但未检测到“完成”关键词,可能失败`);
              return;
            }

            if (hasCompleted) {
              clearInterval(interval);
              this.deployStatus.push(`✅ ${target === 'origin' ? '原模型' : '量化模型'} 评分完成`);
            }
          }
        } catch (error) {
          const isAxiosError = error.isAxiosError;
          const errMsg = error?.message || '';
          const isIgnorable = errMsg.includes('ERR_EMPTY_RESPONSE') ||
              (isAxiosError && !error.response);

          if (isIgnorable) {
            failCount++;
            console.warn(`⚠️ 评分轮询失败(可忽略): ${errMsg},当前失败次数: ${failCount}`);

            if (failCount >= MAX_FAILS) {
              clearInterval(interval);
              this.deployStatus.push(`❌ ${target === 'origin' ? '原模型' : '量化模型'} 连续多次无法获取评分进度,任务可能失败`);
              this.$message.error('评分进度查询连续失败,已中止');
            }
            return;
          }
          clearInterval(interval);
          this.deployStatus.push(`❌ ${target === 'origin' ? '原模型' : '量化模型'} 评分进度获取失败: ${errMsg}`);
          this.$message.error('评分进度查询失败');
        }
      }, 3000);
    },
    \end{lstlisting}
    当流程进行到原模型打分或量化模型评分时调用startEvaluation启动评分程序
    \begin{lstlisting}[language=HTML]
    //评分部分开始
    async startEvaluation(target) {
      const model = this.getCurrentModel();
      const method = this.selectedEvalMethod;

      this.evalLogs = [];
      this.deployStatus.push(`开始对 ${target === 'origin' ? '原模型' : '量化模型'} 进行评分(方法:${method})...`);

      try {
        const response = await axios.post(`${this.apiUrl}`, {
          model_name: model.label,
          eval_method: method,
          eval_tasks: this.selectedEvalTasks,
          start_evaluation: true,
          is_quantized: target !== 'origin'
        }, {
          headers: {
            'Authorization': 'Basic ' + btoa(`${this.authInfo.username}:${this.authInfo.password}`)
          }
        });

        if (response.data.success) {
          this.deployStatus.push(`✅ ${target === 'origin' ? '原模型' : '量化模型'} 评分任务已启动`);
          this.startEvaluationPolling(target);
        } else {
          throw new Error(response.data.message || '评分启动失败');
        }
      } catch (error) {
        console.error(`评分启动失败 (${target})`, error);
        const errorMsg = error.response?.data?.message || error.message;
        this.deployStatus.push(`❌ ${target === 'origin' ? '原模型' : '量化模型'} 评分启动失败: ${error.message}`);
      }
    },
    \end{lstlisting}
    当用户取消部署时,前端向api发送取消信息,后端分别取消对应流程代码
    \begin{lstlisting}[language=HTML]
    //前端向api发送取消信息并监控是否成功取消进程
    async cancelDeploy() {
      try {
        if (!this.isDeploying) return;

        if (this.progressPollingInterval) {
          clearInterval(this.progressPollingInterval);
        }

        this.deployStatus.push('正在取消部署流程...');

        try {
          const quantCancelResp = await axios.post(`${this.apiUrl}/cancel_quant`, {}, {
            headers: {
              'Authorization': 'Basic ' + btoa(`${this.authInfo.username}:${this.authInfo.password}`)
            }
          });

          if (quantCancelResp.data.success) {
            this.deployStatus.push('✅ 已成功取消量化进程');
          } else {
            this.deployStatus.push('⚠️ 取消量化失败: ' + quantCancelResp.data.message);
          }
        } catch (e) {
          this.deployStatus.push('⚠️ 取消量化时发生异常: ' + (e.message));
        }

        try {
          const cancelResp = await axios.post(`${this.apiUrl}/cancel_eval`, {}, {
            headers: {
              'Authorization': 'Basic ' + btoa(`${this.authInfo.username}:${this.authInfo.password}`)
            }
          });

          if (cancelResp.data.success) {
            this.deployStatus.push(`✅ 已取消评分进程`);
          } else {
            this.deployStatus.push(`⚠️ 无法取消评分: ${cancelResp.data.message}`);
          }
        } catch (error) {
          this.deployStatus.push(`⚠️ 取消评分失败: ${error.message}`);
        }

        try {
          const cancelResp = await axios.post(`${this.apiUrl}/cancel_deployment`, {}, {
            headers: {
              'Authorization': 'Basic ' + btoa(`${this.authInfo.username}:${this.authInfo.password}`)
            }
          });

          if (cancelResp.data.success) {
            this.deployStatus.push(`✅ 已取消部署进程`);
            this.$message.success('部署进程已成功取消');
          } else {
            this.deployStatus.push(`⚠️ 无法取消部署: ${cancelResp.data.message}`);
            this.$message.warning(cancelResp.data.message || '无法取消部署进程');
          }
        } catch (error) {
          const errMsg = error?.response?.data?.message || error.message;
          this.deployStatus.push(`⚠️ 取消部署失败: ${errMsg}`);
          this.$message.error('取消部署失败');
        }

        try {
          const cancelResp = await axios.post(`${this.apiUrl}/cancel_compile`, {}, {
            headers: {
              'Authorization': 'Basic ' + btoa(`${this.authInfo.username}:${this.authInfo.password}`)
            }
          });

          if (cancelResp.data.success) {
            this.deployStatus.push(`✅ 已取消编译进程`);
          } else {
            this.deployStatus.push(`⚠️ 无法取消编译: ${cancelResp.data.message}`);
          }
        } catch (error) {
          this.deployStatus.push(`⚠️ 取消编译失败: ${error.message}`);
        }

        this.isDeploying = false;
        this.$message.warning('部署流程和评分流程已中断');

      } catch (error) {
        console.error('取消部署失败:', error);
        const errorMsg = error.response?.data?.message || error.message;
        this.deployStatus.push(`⚠️ 取消失败: ${errorMsg}`);
        this.$message.error(`取消失败: ${errorMsg}`);
      } finally {
        this.isDeploying = false;
      }
    },
    \end{lstlisting}
    \begin{lstlisting}[language=python]
    //取消评分进程
    @app.route('/api/cancel_eval', methods=['POST'])
    @auth.login_required
    def cancel_evaluation():
        global current_eval_process
        
        try:
            if not is_eval_running():
                return jsonify({
                    'success': False,
                    'message': '没有正在运行的评估进程'
                }), 400
            
            with open(EVALUATION_LOG, 'a') as f:
                f.write("[INFO] 正在取消评估进程...\n")
            
            current_eval_process.terminate()
            current_eval_process.join(timeout=2)
            
            with open(EVALUATION_LOG, 'a') as f:
                f.write("[INFO] 评估进程已被用户取消\n")
            
            current_eval_process = None
            
            return jsonify({
                'success': True,
                'message': '评估进程已成功取消'
            })
        except Exception as e:
            error_msg = f"取消评估失败: {str(e)}"
            log_error(error_msg, "backend")
            return jsonify({
                'success': False,
                'message': error_msg
            }), 500
    \end{lstlisting}
    
    \item GPU部署:支持同时部署原模型、量化模型以便对比
    \begin{lstlisting}[language=HTML]
    //根据用户需求进行对应模型部署
    async startDeployment() {
      const model = this.getCurrentModel();

      this.deployLogs = [];
      this.deployStatus.push(`开始部署模型 ${model.label} ...`);

      try {
        const response = await axios.post(`${this.apiUrl}`, {
          model_name: model.label,
          start_deployment: true
        }, {
          headers: {
            'Authorization': 'Basic ' + btoa(`${this.authInfo.username}:${this.authInfo.password}`)
          }
        });

        if (response.data.success) {
          this.deployStatus.push(`✅ 模型 ${model.label} 部署任务已启动`);
          this.startDeploymentPolling();
        } else {
          throw new Error(response.data.message || '部署启动失败');
        }
      } catch (error) {
        console.error('部署启动失败', error);
        const errorMsg = error.response?.data?.message || error.message;
        this.deployStatus.push(`❌ 模型 ${model.label} 部署启动失败: ${errorMsg}`);
      }
    },
    \end{lstlisting}
    用户可在模型部署界面选择对原模型/量化模型进行部署,后续将补充对话框,实现部署后用户即时与模型互动直观体验功能
    \begin{lstlisting}[language=HTML]
    //模型部署界面
    <template>
      <el-card class="deployment-tool-card">
        <div slot="header" class="header-with-icon">
          <img src="@/assets/header-icon.jpg" class="header-icon">
          <span>部署模型管理</span>
        </div>
    
        <el-form label-position="top">
          <!-- 模型名称选择 -->
          <el-form-item label="模型名称">
            <el-select v-model="selectedModel" placeholder="请选择模型">
              <el-option
                  v-for="model in models"
                  :key="model.value"
                  :label="model.label"
                  :value="model.value">
                <span style="float: left">{{ model.label }}</span>
                <img :src="model.icon" class="option-icon" />
              </el-option>
            </el-select>
          </el-form-item>
    
          <!-- 部署类型选择 -->
          <el-form-item label="部署类型">
            <el-select v-model="deployType" placeholder="请选择部署类型">
              <el-option
                  v-for="type in deployTypes"
                  :key="type.value"
                  :label="type.label"
                  :value="type.value">
                <span style="float: left">{{ type.label }}</span>
              </el-option>
            </el-select>
          </el-form-item>
    
          <!-- 启动部署按钮 -->
          <div class="deploy-button-wrapper">
            <el-form-item>
              <el-button type="primary"
                         :loading="isDeploying"
                         :disabled="!selectedModel || !deployType"
                         @click="startDeploy">
                启动部署
              </el-button>
            </el-form-item>
          </div>
    
          <!-- 部署日志展示区域 -->
          <el-card class="deploy-log-card" v-if="deployLogs.length > 0">
            <div class="log-title">部署日志</div>
            <div class="log-content">
              <div v-for="(log, index) in deployLogs" :key="index" class="log-line">{{ log }}</div>
            </div>
          </el-card>
    
        </el-form>
      </el-card>
    </template>
    \end{lstlisting}

    \item API接口:捕捉用户需求并启动后端程序
    \begin{lstlisting}[language=python]
    WORKSPACE_ROOT = "/data/disk0/Workspace/Compiler-Toolchain/Compiler-Toolchain"
    sys.path.insert(0, WORKSPACE_ROOT)
    os.chdir(WORKSPACE_ROOT) 
    //以量化板块为例展示进程启动与日志捕获
    def quantification_entrypoint(model_id, log_path, is_vl_model):
    try:
        gptq_log_dir = os.path.join(WORKSPACE_ROOT, "CT", "WebUI", "gptq_log")
        os.makedirs(gptq_log_dir, exist_ok=True)
        os.chdir(gptq_log_dir)
    
        with open(log_path, 'a') as f:
            with contextlib.redirect_stdout(f), contextlib.redirect_stderr(f):
                if is_vl_model:
                    from CT.Example.Quantization.qwenVLQuantization import simpleQuantization
                    print(f"[INFO] 启动VL模型量化: {model_id}")
                else:
                    from CT.Example.Quantization.quantization import simpleQuantization
                    print(f"[INFO] 启动普通模型量化: {model_id}")
                    
                simpleQuantization(model_id)

                print(f"[INFO] 模型量化完成: {model_id}")

    except Exception as e:
        with open(log_path, 'a') as f:
            f.write(f"[ERROR] 模型量化异常: {e}\n")
    
    def is_quant_running():
        global current_quant_process
        return current_quant_process is not None and current_quant_process.is_alive()
    
    def run_quantification(model_name):
        global current_quant_process
    
        try:
            # 清空进度日志
            with open(PROGRESS_LOG, 'w') as f:
                f.write("")
            
            is_vl_model = "VL" in model_name.upper()
            log_error(f"开始量化模型: {model_name} (类型: {'VL' if is_vl_model else '普通'})", "quant")
    
            # 创建并启动量化进程
            current_quant_process = multiprocessing.Process(
                target=quantification_entrypoint,
                args=(model_name, PROGRESS_LOG, is_vl_model,)
            )
            current_quant_process.start()
    
            return current_quant_process.pid
        
        except Exception as e:
            error_msg = f"量化失败 - 模型:{model_name} 错误:{traceback.format_exc()}"
            log_error(error_msg, "quant")
            current_quant_process = None
            raise
    \end{lstlisting}
    \begin{lstlisting}[language=python]
    //接收前端信息执行对应操作
    @app.route('/api', methods=['POST'])
    @auth.login_required
    def post_api():
        try:
            if data.get("start_quantization"):
            if "model_name" not in data:
                error_msg = "缺少模型名称参数"
                log_error(error_msg, "backend")
                return jsonify({'success': False, 'message': error_msg}), 400
            
            try:
                if is_quant_running():
                    current_quant_process.terminate()
                    time.sleep(1) 
                
                pid = run_quantification(data["model_name"])
                
                log_error(f"已启动量化进程 PID: {pid}", "backend")
                return jsonify({
                    'success': True,
                    'message': '量化进程已启动',
                    'pid': pid,
                    'current_params': {
                        'model_name': data["model_name"]
                    }
                })
            except Exception as e:
                error_msg = f"量化进程启动失败: {str(e)}"
                log_error(error_msg, "backend")
                return jsonify({'success': False, 'message': error_msg}), 500
                
    @app.route('/api/progress', methods=['GET'])
    @auth.login_required
    def get_progress():
        """获取量化进度"""
        try:
            if not os.path.exists(PROGRESS_LOG):
                return jsonify({
                    'success': False, 
                    'message': '进度文件不存在',
                    'is_running': False
                }), 404
            
            with open(PROGRESS_LOG, 'r') as f:
                lines = f.readlines()[-150:]  
            
            # 过滤 ANSI 转义字符
            clean_lines = [remove_ansi_codes(line.strip()) for line in lines if line.strip()]
            
            return jsonify({
                'success': True,
                'progress': clean_lines,
                'is_running': is_quant_running()
            })
        except Exception as e:
            log_error(f"获取进度失败: {str(e)}", "backend")
            return jsonify({
                'success': False, 
                'message': '获取进度失败',
                'is_running': False
            }), 500

    @app.route('/api/cancel_quant', methods=['POST'])
    @auth.login_required
    def cancel_quantization():
        global current_quant_process
        
        try:
            if not is_quant_running():
                return jsonify({
                    'success': False,
                    'message': '没有正在运行的量化进程'
                }), 400
            
            with open(PROGRESS_LOG, 'a') as f:
                f.write("[INFO] 正在取消量化进程...\n")
            
            current_quant_process.terminate()
            current_quant_process.join(timeout=2) 
            
            with open(PROGRESS_LOG, 'a') as f:
                f.write("[INFO] 量化进程已被用户取消\n")
            
            current_quant_process = None
            
            return jsonify({
                'success': True,
                'message': '量化进程已成功取消'
            })
        except Exception as e:
            error_msg = f"取消量化失败: {str(e)}"
            log_error(error_msg, "backend")
            return jsonify({
                'success': False,
                'message': error_msg
            }), 500
    \end{lstlisting}
    
    \item 客制化:支持定制页面图标、企业名称、语言
    前端代码通过组件进行组合,针对客户指定的页面需求可快速调整背景图片、图标、网页风格等一系列元素,实现便捷定制。
    \end{itemize}   
    
\subsection{FPGA OpenAI-Style API}

\section{开发计划表}
\begin{itemize}
    \item 未来会继续向现有的工具链中添加新的功能
    \item 除了在各个环节加入新的功能外,还将额外开发Cli命令行工具以便在服务器终端操作调用
    \item 苏灿同学会同时参与量化部分的llmc开发
\end{itemize}
\begin{figure}[H]
    \centering
    \includegraphics[width=\textwidth]{./figures/Development Schedule.png}
    \caption{开发计划表}
\end{figure}

\end{document}