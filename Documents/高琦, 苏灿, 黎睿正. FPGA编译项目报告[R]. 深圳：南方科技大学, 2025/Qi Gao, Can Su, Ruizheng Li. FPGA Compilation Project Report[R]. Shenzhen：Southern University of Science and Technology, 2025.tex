% !Mode:: "TeX:UTF-8"
% !TEX program  = xelatex
% !BIB program  = biber
\documentclass[AutoFakeBold,AutoFakeSlant,language=chinese,degree=bachelor]{sustechthesis}
\usepackage{pgfplots}
\usepackage{booktabs}
\usepackage{caption}
% !Mode:: "TeX:UTF-8"
% !TEX program  = xelatex

% 数学符号与环境
\usepackage{amsmath,amssymb}
  \newcommand{\dd}{\mathrm{d}}
  \newcommand{\RR}{\mathbb{R}}
% 参考文献
\usepackage[style=gb7714-2015,gbpunctin=false]{biblatex}
  \addbibresource{ref.bib}
% 无意义文本
\usepackage{zhlipsum,lipsum}
% 列表环境设置
\usepackage{enumitem}
% 浮动题不越过 \section
\usepackage[section]{placeins}
% 超链接
\usepackage{hyperref}
% 图片,子图,浮动题设置
\usepackage{graphicx,subcaption,float}
% 抄录环境设置,更多有趣例子请命令行输入 `texdoc tcolorbox`
\usepackage{tcolorbox}
  \tcbuselibrary{xparse}
  \DeclareTotalTCBox{\verbbox}{ O{green} v !O{} }%
    {fontupper=\ttfamily,nobeforeafter,tcbox raise base,%
    arc=0pt,outer arc=0pt,top=0pt,bottom=0pt,left=0mm,%
    right=0mm,leftrule=0pt,rightrule=0pt,toprule=0.3mm,%
    bottomrule=0.3mm,boxsep=0.5mm,bottomrule=0.3mm,boxsep=0.5mm,%
    colback=#1!10!white,colframe=#1!50!black,#3}{#2}%
\tcbuselibrary{listings,breakable}
  \newtcbinputlisting{\Python}[2]{
    listing options={language=Python,numbers=left,numberstyle=\tiny,
      breaklines,commentstyle=\color{white!50!black}\textit},
    title=\texttt{#1},listing only,breakable,
    left=6mm,right=6mm,top=2mm,bottom=2mm,listing file={#2}}
% 三线表支持
\usepackage{booktabs}

% LaTeX logo
\usepackage{hologo}

% !Mode:: "TeX:UTF-8"
% !TEX program  = xelatex
\设置信息{
    % 键 = {{中文值}, {英文值}},
    分类号 = {{}, {}},
    编号 = {{}, {}},
    UDC = {{}, {}},
    密级 = {{}, {}},
    % 仅题目(不含副标题)、系别、专业,支持手动 \\ 换行,不支持自动换行。
    题目 = {{FPGA编译项目报告}, {}},
    % 如无需副标题,删除值内容即可,不可删除键定义。
    副标题 = {{技术报告}, {}},
    姓名 = {{高琦,苏灿,黎睿正}, {Qi Gao, Su Can, Li Ruizheng}},
    学号 = {{12413316}, {12413316}},
    系别 = {{}, {}},
    专业 = {{}, {}},
    指导老师 = {{余浩}, {}},
    时间 = {{}, {}},
    职称 = {{}, {}},
}

\begin{document}
\前序格式化
\摘要标题
\正文格式化

\section{项目背景}
目前大模型部署存在的难点
\begin{itemize}
    \item 工具结合度低
    
        现有的量化、打分和部署等工具模块间甚至模块内操作逻辑不一致,许多操作之间仍需手动调整参数。
    \item 新模型适配成本高
    
        现有的工作流需要根据所使用的模型手动调整,新模型需要逐一环节调试,无法开箱即用。
    \item 操作学习成本高
    
        现有的工具多为命令行工具,缺少直观的交互逻辑和图形化界面,用户操作上手难度较大。
\end{itemize}

\section{项目内容}
\subsection{技术特性}
\begin{itemize}
    \item 量化:支持GPTQ指定bits、group\_size、desc\_act
    \item 打分
        \begin{itemize}
            \item 支持lm-evaluation-harness测试项
            
                arc\_easy、arc\_challenge、gsm8k\_cot、gsm8k\_platinum\_cot、hellaswag、mmlu、gpqa、boolq、openbookqa
            \item 支持EvalPlus测试项
            
                humaneval、mbpp
        \end{itemize}

    \item GPU部署:支持vLLM指定上下文长度、显存占用限制、服务端口、API密钥
    \item 权重处理:支持Compiler-VCU128指定bits、group\_size、desc\_act
    \item FPGA服务:支持Fast API指定上下文长度、生成温度、服务端口、API密钥
    \item GPU、Port调度器:支持GPU设备图分类、调整调度显存占用量、设备锁机制
    \item WebUI:支持指定服务地址、用户管理
\end{itemize}

\subsection{WebUI}
\begin{itemize}
    \item 用户管理:支持自定义登录用户
    \item 打分:支持同时开展多框架多测试项测试和原模型、量化模型对比测试
    \item GPU部署:支持同时部署原模型、量化模型以便对比
    \item 客制化:支持定制页面图标、企业名称、语言
\end{itemize}

\subsection{FPGA OpenAI-Style API}

\section{开发计划表}
\begin{itemize}
    \item 未来会继续向现有的工具链中添加新的功能
    \item 除了在各个环节加入新的功能外,还将额外开发Cli命令行工具以便在服务器终端操作调用
    \item 苏灿同学会同时参与量化部分的llmc开发
\end{itemize}
\begin{figure}[H]
    \centering
    \includegraphics[width=.64\textwidth]{./figures/Development Schedule.png}
    \caption{开发计划表}
\end{figure}

\end{document}